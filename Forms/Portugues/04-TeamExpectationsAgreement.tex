\documentclass{letter}
\usepackage{color}
\usepackage{xcolor}
\usepackage{url}
\usepackage{graphicx}
\usepackage{fancyhdr}
\usepackage[utf8]{inputenc}
\usepackage[T1]{fontenc}

\signature{\ }
\address{\ \\
EEE933 - Planejamento e Análise de Experimentos\\
Prof. Felipe Campelo}

\fancypagestyle{firstpage}{\fancyhf{}\fancyhead[R]{\includegraphics[height=.8in, keepaspectratio=true]{../../figs/ENG_ELETRICA_ufmg.png}}}
\fancypagestyle{empty}{\fancyhf{}\fancyhead[R]{\includegraphics[height=.8in, keepaspectratio=true]{../../figs/ENG_ELETRICA_ufmg.png}}}

\begin{document}
\pagestyle{empty}
\begin{letter}{}
\date{}
\opening{\vskip -5em}

\begin{center}
{\sc Acordo sobre as expectativas da equipe}
\end{center}

\noindent \textbf{Nome da equipe}: 

\hrulefill{}

\noindent \textbf{Integrantes}: 

\noindent 1. \hrulefill{}

\noindent 2. \hrulefill{}

\noindent 3. \hrulefill{}

\noindent 4. \hrulefill{}

\noindent \textbf{Coloquem neste documento as regras e expectativas que a equipe
concorda em adotar. Vocês podem lidar com quaisquer aspectos das responsabilidades
necessárias para um bom funcionamento do time: preparação para a aula;
participação de reuniões extra-classe; ter certeza que todos entendam
todas as soluções propostas; comunicação franca, mas com respeito,
quando surgirem conflitos, etc.. Cada membro da equipe deve assinar esse
acordo, o que indicará a aceitação dessas expectativas e intenção
de cumpri-las. Encaminhem uma cópia ao professor e fiquem com a outra cópia para o registro de vocês.}

\noindent \emph{Esse acordo é para que vocês possam utilizá-lo em benefício próprio. Ele não será corrigido ou receberá comentários, a não
ser que vocês o peçam explicitamente.} Notem, contudo, que se vocês fizerem uma lista suficientemente completa porém \textbf{realista}, vocês maximizarão a chance de sucesso. Por exemplo, \textquotedbl{}Cada membro vai resolver
individualmente todos problemas de cada trabalho antes de nos reunirmos\textquotedbl{}
ou \textquotedbl{}Vamos tirar a nota máxima em todas as tarefas\textquotedbl{}
ou \textquotedbl{}Nós jamais desmarcaremos um encontro preparatório extra-classe\textquotedbl{} provavelmente não são expectativas realistas,
mas \textquotedbl{}Vamos todos nos preparar para as aulas e para as tarefas a serem desenvolvidas\textquotedbl{} e \textquotedbl{}Vamos todos nos esforçar para que aqueles que não conseguiram se preparar para as atividades consigam fazê-las em aula\textquotedbl{} são afirmações mais realistas.

\newpage
\end{letter}
\end{document}
