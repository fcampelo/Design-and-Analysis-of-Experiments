\documentclass{letter}
\usepackage{color}
\usepackage{xcolor}
\usepackage{url}
\usepackage{graphicx}
\usepackage{fancyhdr}
\usepackage[utf8]{inputenc}
\usepackage[T1]{fontenc}

\signature{\ }
\address{\ \\
EEE933 - Planejamento e Análise de Experimentos\\
Prof. Felipe Campelo}

\fancypagestyle{firstpage}{\fancyhf{}\fancyhead[R]{\includegraphics[height=.8in, keepaspectratio=true]{../../figs/ENG_ELETRICA_ufmg.png}}}
\fancypagestyle{empty}{\fancyhf{}\fancyhead[R]{\includegraphics[height=.8in, keepaspectratio=true]{../../figs/ENG_ELETRICA_ufmg.png}}}

\begin{document}
\pagestyle{empty}
\begin{letter}{}
\date{}
\opening{\vskip -5em}

\begin{center}
{\sc Declaração de Políticas de Equipe\footnote{Adaptado de B. Oakley \textit{et al.}, ``Turning Student Groups into Effective Teams'', 2004.}}
\end{center}

Cada equipe será incumbida de determinadas responsabilidades na condução e realização de problemas e projetos. O presente documento apresenta as políticas de equipe que devem ser seguidas no âmbito da disciplina.

\begin{itemize}
\item Para cada trabalho, determinem um membro como \textit{Coordenador}, um como \textit{Relator}, e um como \textit{Verificador} do trabalho. Em times compostos por quatro membros, determinem também um membro como \textit{Monitor}. Estes papéis devem ser alternados a cada trabalho;
\item Entrem em acordo sobre um horário comum para reuniões, e deixem claro as tarefas que cada um deve realizar até a reunião (leituras, trabalho preliminar em algum aspecto técnico, etc.). Os membros da equipe devem completar seus preparativos individuais antes de cada reunião;
\item O \textit{Coordenador} deve entrar em contato com os demais membros antes de cada reunião, se certificando de que todos estejam cientes do local e horário do encontro, bem como das tarefas alocadas a cada um;
\item \textbf{\textit{Reunião e trabalho}}: 
\begin{itemize}
\item O papel do \textit{Coordenador} é manter o grupo focado, e se certificar de que todos estão envolvidos no trabalho;
\item O \textit{Relator} deve trabalhar no preparo da versão final do trabalho a ser entregue;
\item O \textit{Verificador} realiza a verificação final do trabalho antes que o mesmo seja entregue;
\item O \textit{Monitor} é responsável por se certificar que todos entendem tanto a solução encontrada quanto a estratégia utilizada para encontrá-la. Em equipes compostas por três membros os papéis de \textit{Monitor} e \textit{Verificador} devem ser desempenhados pelo mesmo membro;
\end{itemize} 
Ao final de cada encontro o grupo deve agendar a próxima reunião (data/local) e os papéis que cada um irá desempenhar no trabalho seguinte;
\item O \textit{Verificador} é responsável pela entrega da versão final do trabalho, com os nomes de todos os membros \textit{que participaram ativamente no mesmo}. Se o \textit{Verificador} tiver problemas de agenda e não puder comparecer à aula no dia e horário de entrega, esta tarefa pode ser delegada a outro membro da equipe.
\item É importante revisar as atividades corrigidas, e se certificar de que cada um entenda por quê pontos foram perdidos, e como corrigir estes erros.
\item \textit{Entrem em contato com o professor caso haja algum conflito que não possa ser resolvido pela própria equipe;}
\item \textbf{Como lidar com membros não-cooperativos}: 
\begin{itemize}
\item Se algum membro da equipe se recusa a cooperar em suas tarefas (por qualquer que seja o motivo), seu nome não deve ser incluído no trabalho entregue;
\item Se o problema persistir, a equipe deve agendar um encontro com o professor de forma a tentar resolver o problema;
\item Caso não seja possível encontrar uma boa solução e o problema ainda persistir, o restante da equipe deve notificar o membro não-cooperativo (por escrito, com cópia para o professor) que ele está sob risco de ser excluído do grupo.
\item Caso as tentativas anteriores tenham sido em vão, o grupo deve então notificar o membro não-cooperativo (por escrito, com cópia para o professor) que o mesmo está excluído do grupo;
\item Alunos excluídos de alguma equipe possuem duas alternativas: 
\begin{itemize}
\item Encontrar um grupo que conte com somente três membros e que esteja disposto a incorporá-lo na equipe (neste caso os três membros da equipe devem se manifestar por escrito ao professor); 
\item Realizar os trabalhos restantes de forma individual.
\end{itemize}
\item Da mesma forma, se algum dos membros da equipe julgar que está fazendo todo o trabalho do grupo, o mesmo deve comunicar por escrito ao grupo que pretende se desligar do grupo caso não haja uma maior cooperação, e um segundo comunicado por escrito declarando seu desligamento da equipe caso o primeiro não resulte em uma maior participação dos demais membros. Assim como no caso anterior, todos os comunicados devem ser feitos por escrito com cópia para o professor. As opções para o membro da equipe que se desligar de um grupo são as mesmas detalhadas anteriormente: encontrar um grupo que o receba ou realizar as tarefas individualmente.
\end{itemize} 
\end{itemize}

Trabalhar em equipes nem sempre é uma tarefa simples -- membros da equipe por vezes tem outras responsabilidades e não conseguem preparar suas parcelas do trabalho ou participar das reuniões, e conflitos resultantes de níveis distintos de habilidade ou compreensão do conteúdo podem ocorrer, bem como de aspectos ligados à ética de trabalho. Entretanto, em equipes que trabalham e se comunicam bem os benefícios superam em muito as dificuldades. 

Uma das formas de melhorar as chances de uma dada equipe funcionar bem consiste em determinar conjuntamente, e antes do início das atividades, quais são as expectativas coletivas dos membros da equipe em relação aos colegas. E este é o objetivo da primeira tarefa de cada equipe, a saber, a definição de um \textit{Acordo de Expectativas da Equipe}.

\end{letter}
\end{document}
