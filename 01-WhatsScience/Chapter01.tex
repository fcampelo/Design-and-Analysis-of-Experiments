\documentclass[t]{beamer}

% Load general definitions
% Preamble file - general definitions, package loading, etc.

%=================================
% Load packages
\usepackage{amssymb,amsmath}
\usepackage{graphicx}
\usepackage{url}
\usepackage{tikz}
\usetikzlibrary{mindmap,trees,arrows}
\usepackage{fancyvrb}
\usepackage[english]{babel}
\usepackage[latin1]{inputenc}
\usepackage{subfigure}
\usepackage{times}
\usepackage[T1]{fontenc}
\usepackage{cancel}
\usepackage{color}
\usepackage{listings}

%=================================
% Set mode
\mode<presentation>
{
	\usetheme{Madrid}
	\usecolortheme{whale}
	\useoutertheme{infolines}
	\setbeamercovered{invisible}
}

% Get rid of nav bar
\beamertemplatenavigationsymbolsempty

% Insert frame number at bottom of the page.
\usefoottemplate{\hfil\tiny{\color{black!90}\insertframenumber}} 

%=================================
% Define new commands

\newcommand\Real{{\mathbb{R}}}
%\newcommand{\vi}{\vspace{0.6\baselineskip}}
%\newcommand{\goodgap}{\hspace{\subfigtopskip}\hspace{\subfigbottomskip}}


% Equation environments
\newcommand{\beq}{\begin{equation}}
\newcommand{\eq}{\end{equation}}
\newcommand{\beqs}{\begin{equation*}}
\newcommand{\eqs}{\end{equation*}}
\newcommand{\beqn}{\begin{eqnarray}}
\newcommand{\eqn}{\end{eqnarray}}

% Bold variables
\newcommand{\mbf}[1]{\ensuremath{\mathbf{#1}}}

% Itemization
\newcommand{\bitem}{\begin{itemize}}
\newcommand{\eitem}{\end{itemize}}
\newcommand{\spitem}{\vskip 1em\item}
\newcommand{\bitems}{\begin{itemize}\item}
\newcommand{\benums}{\begin{enumerate}\item}
\newcommand{\eenum}{\end{enumerate}}

% color blocks
\newenvironment{colorblock}[2]{%
\setbeamercolor{block title}{#2}
\begin{block}{#1}}{\end{block}}

% Vertical spacing
\newcommand{\vone}{\vskip 1em}
\newcommand{\vhalf}{\vskip .5em}

% Frame environments
\newenvironment{ftst}[3][t]{%
\begin{frame}{environment=ftst,#1}
\frametitle{#2}
\framesubtitle{#3}}{\end{frame}}

\newenvironment{ftstf}[2]{
\begin{frame}[fragile,environment=ftstf]
\frametitle{#1}
\framesubtitle{#2}}{\end{frame}}

% colors
\definecolor{MyGray}{rgb}{0.5,0.5,0.5}
\definecolor{MyDBGray}{rgb}{0.1,0.1,0.4}
\definecolor{darkgreen}{rgb}{0,0.4,0}
\definecolor{black}{rgb}{0,0,0}
\def\defn#1{{\color{red} #1}}

% Footnote
\renewcommand{\thefootnote}{\alph{footnote}}

% Relaxed footnotes
\newcommand{\lfr}[1]{\let\thefootnote\relax\footnote{\tiny #1}}

% Verbatim environment - using FANCYVRB package
%\DefineVerbatimEnvironment%
%{rcode}{Verbatim}
%{fontsize=\scriptsize}

% Verbatim environment - using LISTINGS package
\lstnewenvironment{rcode} {\lstset{ language = R,
                                                        basicstyle = \tiny\ttfamily,
                                                        showspaces = false,
                                                        showstringspaces = false,
                                                        showtabs = false,
                                                        keywordstyle = \color{black}\bfseries,
                                                        commentstyle = \color{darkgreen}\bfseries,
                                                        numbers = none,
                                                        otherkeywords={<-,<<-},
                                                        deletekeywords={data,model}}}%
{}


% Specific definitions
\title[]{Design and Analysis of Experiments}
\subtitle[]{01 - What is Science}
\author[]{Felipe Campelo\\{\footnotesize http://www.cpdee.ufmg.br/\textasciitilde fcampelo}}
\institute{Graduate Program in Electrical Engineering}
\date{\scriptsize Belo Horizonte\\March 2015}

\begin{document}

% cover page
\setbeamertemplate{footline}{}
\begin{frame}
\begin{flushright}
\includegraphics[width=.25\textwidth]{../figs/principal_completa3_ufmg}
\end{flushright}
  \titlepage
  \begin{tikzpicture}[remember picture,overlay]
  \node[anchor=south east,xshift=-5pt,yshift=122pt] at (current page.south east) {\tiny Version 2.11};
  \node[anchor=south west,yshift=0pt] at (current page.south west) {\includegraphics[width=.15\textwidth]{../figs/by-nc-sa.png}};
  \end{tikzpicture}  
\end{frame}

%=====

% quotation page
  \begin{frame}[b]
		\frametitle{}
\begin{columns}[T]
\column{0.8\textwidth}
\flushright{\small ``\textit{Somewhere, something incredible\\is waiting to be known.}''\\\ \\
Carl E. Sagan (1934 -- 1996)\\
American astronomer}
\column{0.2\textwidth}
\begin{tikzpicture}[remember picture,overlay]
\node[anchor=south east,yshift=5pt,xshift=0pt] at (current page.south east)
{\includegraphics[width=\textwidth]{../figs/sagan.png}};
\end{tikzpicture}
\end{columns}
\vhalf
\let\thefootnote\relax\footnote{\tiny Image: \url{http://www.relativelyinteresting.com/miss-carl-sagan/}}
\end{frame}

%=====

% Main slides
\begin{ftst}
{What is science?}
{Some common misconceptions}

\bitem\item Science is a collection of facts; {\color{red}$\times$}
	\item Science is the creation of new gadgets; {\color{red}$\times$}
	\item Scientific ideas are absolute and unchangeable; {\color{red}$\times$}
	\item Scientific ideas are subject to change, therefore unreliable; {\color{red}$\times$}
	\item Observations give answers directly to the scientists; {\color{red}$\times$}
	\item Science \textbf{proves} stuff; {\color{red}$\times$}
	\item Science can only \textbf{disprove} stuff; {\color{red}$\times$}
	\item The scientist works to \textbf{show} that his/her theory is right;{\color{red}$\times$}
	\item \textbf{Theory} \textit{versus} \textbf{Law}.
\eitem
\let\thefootnote\relax\footnote{\tiny\textbf{Essential reading}: Common Misconceptions About Science: \url{http://goo.gl/TN7k9B}}
\let\thefootnote\relax\footnote{\tiny Image: \url{http://xkcd.com}}
\begin{tikzpicture}[remember picture,overlay]
\node[anchor=north east,yshift=-35pt,xshift=-5pt] at (current page.north east) { \includegraphics[height=2cm]{../figs/xkcd_science.jpg}};
\end{tikzpicture}
\end{ftst}

%=====

\begin{ftst}
{What is science?}
{A good operational definition}
\vone
\begin{columns}[T]
\column{0.3\textwidth}
\begin{tikzpicture}[remember picture,overlay]
\node[anchor=north west,yshift=-70pt,xshift=0pt] at (current page.north west)
{\includegraphics[width=\textwidth]{../figs/StevenNovella.jpg}};
\end{tikzpicture}
\column{0.7\textwidth}
\begin{colorblock}{}{bg=green!30,fg=black}
\flushleft{``\textit{What do you think science is?\\
There's nothing magical about science.\\
It is simply a systematic way for carefully\\
and thoroughly observing nature and\\
using consistent logic to evaluate results.}''}
\flushright{\vskip -1em -- Steven P. Novella}
\end{colorblock}
\end{columns}
\vhalf
\let\thefootnote\relax\footnote{\tiny Image: \url{http://www.relativelyinteresting.com/definition-science-steven-novella/}}
\end{ftst}

%=====

\begin{ftst}
{What is science?}
{The scientific process}
\begin{columns}[T]
	\column{0.6\textwidth}
		\bitems Normally shown as a flowchart or a sequence of steps;
			\item Oversimplification of a complex and iterative process;
			\item Suggests an ``end'' to the process.
		\eitem
	\column{0.4\textwidth} \includegraphics[width=\textwidth]{../figs/sciproc01.jpg}
\end{columns}
\vone
\begin{columns}[T]
	\column{1.02\textwidth}
		\bitems Actually includes:
			\bitems Several activities, performed at different stages;
			\item Interaction with the scientific community;
			\item Creative, ``outside the box'' thinking;
			\item Preliminary conclusions, subject to revision as new and better data become available;
			\item Learning from failures as much as from successes.
		\eitem
	\eitem
\end{columns}
\let\thefootnote\relax\footnote{\tiny Image: \url{http://goo.gl/7cCGaz} - (c) Understanding Science, 2015. Used with permission.}
\end{ftst}

%=====

\begin{ftst}
{What is science?}
{The scientific process}
\begin{center}
	\vspace{-1.3em}
	\includegraphics[width=0.56\textwidth]{../figs/sciproc02.png}
\end{center}
\let\thefootnote\relax\footnote{\tiny Image: \url{http://goo.gl/VglXc5} - (c) Understanding Science, 2015. Used with permission.}
\end{ftst}

%=====

\begin{ftst}
{What is science?}
{The scientific process}
\vspace{-1em}
\begin{colorblock}{}{bg=green!30,fg=black}
``\textit{Dans les champs de l'observation le hasard ne favorise que les esprits pr�par�s.}'' -- \textbf{Louis Pasteur} (Univ. Lille, France, 1854).
\end{colorblock}
\vskip -0.5em
\begin{columns}[T]
\column{0.5\textwidth}
	\bitems Observations $\rightarrow$ \textbf{questions};
	\item Exploratory experimentation;
	\item Preparation + serendipity.
\eitem
\column{0.5\textwidth} \includegraphics[width=\textwidth]{../figs/sciproc02a.png}
\end{columns}
\vhalf
\begin{columns}[T]
\column{0.3\textwidth}
\begin{block}{\footnotesize Benzene (1865)}
\centering\includegraphics[width=0.35\textwidth]{../figs/kekule.png}\\
\footnotesize Kekule
\end{block}
\column{0.02\textwidth}
\column{0.3\textwidth}
\begin{block}{\footnotesize Radioactivity (1896)}
\centering\includegraphics[width=0.35\textwidth]{../figs/becquerel.png}\\\footnotesize Becquerel
\end{block}
\column{0.02\textwidth}
\column{0.3\textwidth}
\begin{block}{\footnotesize Penicillin (1928)}
\centering\includegraphics[width=0.35\textwidth]{../figs/fleming.png}\\\footnotesize Fleming
\end{block}
\column{0.01\textwidth}
\end{columns}
\let\thefootnote\relax\footnote{\tiny Top image: \url{http://goo.gl/fy8Glh} - (c) Understanding Science, 2015. Used with permission.}
\let\thefootnote\relax\footnote{\tiny Scientists: \url{http://goo.gl/SG6sgp} | \url{http://goo.gl/rhLC9C} | \url{http://goo.gl/CFj8Ml}}
\end{ftst}

%=====

\begin{ftst}
{What is science?}
{The scientific process}
\bitems Drawing and testing hypotheses;
	\item Comparing alternative explanations;
	\item Accepting / rejecting ideas based on \textbf{evidence};
	\item \textbf{Predictions} \textit{versus} \textbf{observation}: corroboration or refutation?
\eitem

\begin{tikzpicture}[remember picture,overlay]
\node[anchor=south,yshift=15, xshift=20pt] at (current page.south) {\includegraphics[width=0.55\textwidth]{../figs/sciproc02b.png}};
\end{tikzpicture}
\let\thefootnote\relax\footnote{\tiny Image: \url{http://goo.gl/aOgSqT} - (c) Understanding Science, 2015. Used with permission.}
\end{ftst}

%=====

\begin{ftst}
{What is science?}
{The scientific process}
\vspace{-0.4em}
\begin{columns}[T]
\column{0.8\textwidth}
	\textbf{James Lind} (1747):\\
	\bitems Observation: scurvy in sailors;
		\item Conjecture: Caused by the body rottenning;
		\item Idea: attempt to avoid/reverse effects with acidic substances;
	\eitem
\column{0.2\textwidth}
\end{columns}
\vone
Separation of a group of 12 affected sailors in six groups with identical diets, except for the addition of a supplement:
\begin{columns}[T]
\column{0.30\textwidth}
\begin{colorblock}{Group 1}{bg=green!25,fg=black}
	\small Cider.
\end{colorblock}
\begin{colorblock}{Group 4}{bg=gray!25,fg=black}
	\small Sea water.
\end{colorblock}
\column{0.30\textwidth}
\begin{colorblock}{Group 2}{bg=gray!25,fg=black}
	\small Vitriol.
\end{colorblock}
\begin{colorblock}{Group 5}{bg=green!60,fg=black}
	\small Oranges and lemons.
\end{colorblock}
\column{0.30\textwidth}
\begin{colorblock}{Group 3}{bg=gray!25,fg=black}
	\small Vinegar.
\end{colorblock}
\begin{colorblock}{Group 6}{bg=gray!25,fg=black}
	\small Tea.
\end{colorblock}
\end{columns}
\begin{tikzpicture}[remember picture,overlay]
\node[anchor=north east,yshift=-40, xshift=-10pt] at (current page.north east) {\includegraphics[width=2.2cm]{../figs/lind.jpg}};
\end{tikzpicture}
\let\thefootnote\relax\footnote{\tiny Image: \url{http://commons.wikimedia.org/wiki/File:James_Lind_by_Chalmers.jpg}}
\end{ftst}

%=====

\begin{ftst}
{What is science?}
{The scientific process}
\begin{columns}[T]
\column{0.5\textwidth}
Interaction with the scientific community is \textbf{fundamental}:

	\bitems Colleagues;
	\item Collaborators;
	\item Reviewers;
	\item Rivals;
\eitem
\column{0.5\textwidth} \vskip -1em\includegraphics[width=1\textwidth]{../figs/sciproc02c.png}
\end{columns}
\vone
This interaction plays essential roles for the progress of research:
\vskip -.5em
\begin{columns}[T]
\column{0.22\textwidth}\begin{block}{Criticism}
	\centering\includegraphics[width=\textwidth]{../figs/community2.png}
\end{block}
\column{0.22\textwidth}\begin{block}{Inspiration}
	\centering\includegraphics[width=\textwidth]{../figs/community3.png}
\end{block}
\column{0.22\textwidth}\begin{block}{Vigilance}
	\centering\includegraphics[width=\textwidth]{../figs/community4.png}
\end{block}
\column{0.22\textwidth}\begin{block}{Motivation}
	\centering\includegraphics[width=\textwidth]{../figs/community5.png}
\end{block}
\end{columns}
\let\thefootnote\relax\footnote{\tiny All images: \url{http://goo.gl/9pSCTG} - (c) Understanding Science, 2015. Used with permission.}
\end{ftst}

%=====

\begin{ftst}
{What is science?}
{The scientific process}
Publication and peer review.
\begin{columns}[T]
\column{0.5\textwidth}
\includegraphics[width=1.15\textwidth]{../figs/peerreview.png}
\column{0.5\textwidth}
\bitems Additionally,  \textit{post-publication review} by the wider scientific community;
\spitem \textbf{Replication} and verification of results;
\spitem \textbf{Reproducibility} is essential.
\eitem
\end{columns}
\let\thefootnote\relax\footnote{\tiny Image: \url{http://goo.gl/VWCVkK} - (c) Understanding Science, 2015. Used with permission.}
\end{ftst}

%=====

\begin{ftst}
{What is science?}
{The scientific process}
\begin{columns}[T]
\column{0.55\textwidth}
The scientific process is a way of building knowledge:
\bitems Generate and test new ideas about how the world works;
	\item Iteratively increasing the reliability of the knowledge;
\eitem
\column{0.45\textwidth}\vskip -1em\includegraphics[width=\textwidth]{../figs/sciproc02d.png}
\end{columns}
\vhalf
\begin{columns}[T]
\column{0.32\textwidth}\begin{block}{}
		\centering\includegraphics[width=0.85\textwidth]{../figs/benefitschart1.png}\\
		\scriptsize Knowledge$\rightarrow$Applications
	\end{block}
\column{0.32\textwidth}\begin{block}{}
		\centering\includegraphics[width=0.83\textwidth]{../figs/benefitschart2.png}\\
		\scriptsize Technologies$\rightarrow$Discovery
	\end{block}
\column{0.32\textwidth}\begin{block}{}
		\centering\includegraphics[width=0.79\textwidth]{../figs/benefitschart3.png}\\
		\scriptsize Applications$\rightarrow$Investigation
	\end{block}
\end{columns}
\let\thefootnote\relax\footnote{\tiny All images: \url{http://goo.gl/IBRSoQ} - (c) Understanding Science, 2015. Used with permission.}
\end{ftst}

%=====

\begin{ftst}
{What is science?}
{To wrap it up}
\vone
\begin{columns}[T]
\column{0.18\textwidth}
\begin{tikzpicture}[remember picture,overlay]
\node[anchor=north west,yshift=-70pt,xshift=0pt] at (current page.north west)
{\includegraphics[width=\textwidth]{../figs/caranha.png}};
\end{tikzpicture}
\column{0.82\textwidth}
\begin{colorblock}{}{bg=green!30,fg=black}
\flushleft{``\textit{It is important to be literate in the scientific method,\\
	not only for the sake of your own research. We are also\\
	agents of change in the population and, as such, we\\
	need to be aware of good and bad science, and able\\
	to point the difference to the society.}''}
\flushright{\vskip -1em -- Claus C. Aranha}
\end{colorblock}
\end{columns}
\vhalf
\let\thefootnote\relax\footnote{\tiny Image: \url{http://lattes.cnpq.br/2897895256340893}}
\end{ftst}

%=====

\begin{ftst}
{Bibliography}
{\ }
\scriptsize
\textbf{Required reading}

\benums \textit{Understanding Science}. 2014. University of California Museum of Paleontology. 3 January 2014. - 
{\tiny \url{http://www.understandingscience.org}}
	\item F.L.H. Wolfs, \textit{APPENDIX E: Introduction to the Scientific Method}. - 
	{\tiny \url{http://goo.gl/osGpU}}
\eenum

\textbf{Recommended reading}

\benums Carl Sagan,\textit{The demon-haunted world: science as a candle in the dark},\\Random House, 1996.
	\item The Skeptics Guide to the Universe. - 
	{\tiny \url{http://www.theskepticsguide.org}}
\eenum
\end{ftst}

%=====

\begin{ftstf}{About this material}{Conditions of use and referencing}
\centering\footnotesize This work is licensed under the Creative Commons CC BY-NC-SA 4.0 license\\(Attribution Non-Commercial Share Alike International License version 4.0).\\
\vhalf
\url{http://creativecommons.org/licenses/by-nc-sa/4.0/}\\
\vone
\footnotesize Please reference this work as:\\
\footnotesize \flushleft Felipe Campelo (2015), \textit{Lecture Notes on Design and Analysis of Experiments}.\\Online: {\scriptsize\url{https://github.com/fcampelo/Design-and-Analysis-of-Experiments}}\\
Version 2.11, Chapter 1; Creative Commons BY-NC-SA 4.0.\\

\begin{Verbatim}[fontsize=\tiny]
    @Misc{Campelo2015-01,
      title={Lecture Notes on Design and Analysis of Experiments},
      author={Felipe Campelo},
      howPublished={\url{https://github.com/fcampelo/Design-and-Analysis-of-Experiments}},
      year={2015},
      note={Version 2.11, Chapter 1; Creative Commons BY-NC-SA 4.0.},
    }
\end{Verbatim}

\begin{tikzpicture} [remember picture,overlay]
\node[anchor=south,yshift=0pt] at (current page.south){ \includegraphics[width=.2\textwidth]{../figs/CCSomerights.png}};
\end{tikzpicture}
\end{ftstf}


\end{document}